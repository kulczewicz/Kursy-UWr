\documentclass[a4paper,12pt]{article}
\usepackage{polski}
\usepackage[utf8]{inputenc}
\usepackage[left = 3cm, right = 3cm, top = 2cm, bottom = 2cm]{geometry}
\usepackage{enumerate}
\usepackage{amssymb}		% pakiet do symboli
\usepackage{mathtools}		% pakiet do matmy (rozszerza amsmath)
\usepackage{enumitem}		% punktowanie (a), (b), ...
\usepackage{nopageno}		% brak numerow stron
\usepackage{graphicx}		% wstawianie obrazkow
\usepackage{float}			% wstawianie obrazkow w dowolnym miejscu
\usepackage{caption}
\usepackage{esdiff}         % pochodne \diff{}{}
\usepackage{listings}
\usepackage{xcolor}
\usepackage{adjustbox}
\usepackage[none]{hyphenat} % usunięcie łamania wyrazów na końcu linii

% nowe komendy dla wygodniejszego pisania :)

\newcommand{\floor}[1]{\left\lfloor #1 \right\rfloor}	% podłoga
\newcommand{\ceil}[1]{\left\lceil #1 \right\rceil}		% sufit
\newcommand{\fractional}[1]{\left\{ #1 \right\}}		% część ułamkowa {x}
\newcommand{\abs}[1]{\left| #1 \right|}					% wartosc bezwzgledna / moc
\newcommand{\set}[1]{\left \{ #1 \right \}}				% zbiór elementów {a,b,c}
\newcommand{\pair}[1]{\left( #1 \right)}				% para elementów (a,b)
\newcommand{\Mod}[1]{\ \mathrm{mod\ #1}}				% lekko zmodyfikowane modulo
\newcommand{\comp}[1]{\overline{ #1 }} 					% dopełnienie zbioru 
\newcommand{\annihilator}{\mathbf{E}}					% operator E
\newcommand{\seqAnnihilator}[1]{\annihilator \left\langle #1 \right\rangle} % E(a_n)
\newcommand{\sequence}[1]{\left\langle #1 \right\rangle} % <a_n>
\DeclareMathOperator{\lcm}{lcm}							% obsługa lcm w mathmode

% styl do kodu
\lstdefinestyle{code}{%
basicstyle=\ttfamily\small,
commentstyle=\color{green!60!black},
keywordstyle=\color{magenta},
stringstyle=\color{blue!50!red},
showstringspaces=false,
numbers=left,
numberstyle=\footnotesize\color{gray},
numbersep=10pt,
tabsize=4,
rulecolor=\color{red},
breaklines=true
}

\newcommand{\code}[1]{\lstinline[style=code]{#1}} % kod inline

\begin{document}
\noindent \textbf{Matematyka dyskretna L, Lista 10 - Tomasz Woszczyński}\newline

\noindent \newline \textbf{Zadanie 1} \newline
Przypuśćmy, że w grafie $G$ wszystkie wagi krawędzi są różne. Pokaż, nie używając
żadnego algorytmu, że $G$ zawiera tylko jedno minimalne drzewo rozpinające. \\

\noindent Załóżmy, że dla $G$ istnieją dwa minimane drzewa rozpinające $T_1$
oraz $T_2$. W jednym z tych drzew musi się znaleźć krawędź o minimalnej wadze
spośród wszystkich wag. Bez straty ogólności, niech ta krawędź będzie w drzewie
$T_1$ i oznaczmy ją przez $e_1$. Wtedy drzewo $T_2 \cup \{ e_1 \}$ zawiera cykl 
i jedna z krawędzi tego cyklu, nazwijmy ją $e_2$, nie należy do $T_1$. Wiemy, że 
krawędzie $e_1$ i $e_2$ są różne i każda z nich należy dokładnie do jednego 
z drzew $T_1$ lub $T_2$. Oznacza to, że $w(e_1) < w(e_2)$, gdyż $e_1$ jest
najlżejszą wagą w drzewie $T_1$, a $e_2$ najlżejszą w drzewie $T_2$ (gdyby tak
nie było, to byśmy zaprzeczyli temu, że każda z tych krawędzi należy tylko
do jednego minimalnego drzewa rozpinającego). Skoro tak, to 
$T = T_2 \cup \{ e_1 \} \setminus \{ e_2 \}$ jest drzewem rozpinającym. Łączna
suma wag drzewa $T$ jest mniejsza od łącznej sumy wag drzewa $T_2$, więc 
dochodzimy do sprzeczności, gdyż założyliśmy, że $T_2$ jest minimalnym
drzewem rozpinającym, co kończy dowód.

\noindent \newline \textbf{Zadanie 2} \newline
Niech $T$ będzie MST grafu $G$. Pokaż, że dla dowolnego cyklu $C$ grafu $G$
drzewo $T$ nie zawiera jakiejś najcięższej krawędzi z $C$. \\

\noindent Załóżmy nie wprost, że najcięższa krawędź $e$ należy do $T$ będącego
minimalnym drzewem rozpinającym grafu $G$. Usunięcie tej krawędzi sprawiłoby,
że $T$ rozpadłoby się na dwa poddrzewa, których liśćmi byłyby wierzchołki 
tej krawędzi. Pozostała część cyklu $C$ może połączyć ponownie te poddrzewa,
więc musi istnieć jakaś krawędź $e_{new} \in C$ oparta na innych wierzchołkach. 
Oznacza to, że krawędź $e_{new}$ łączy poddrzewa w drzewo $T_2$ o wagach 
mniejszych niż te, które znajdowały się w drzewie $T$, ponieważ 
$w(e_{new}) < w(e)$. Dochodzimy więc do sprzeczności, więc MST grafu $G$ nie
może zawierać najcięższej krawędzi cyklu $C$, co kończy dowód.

\newpage
\noindent \textbf{Zadanie 5} \newline
Załóżmy, że wszystkie krawędzie w grafie mają różne wagi. Udowodnij, że 
algorytm Boruvki rzeczywiście znajduje drzewo rozpinające, tzn. pokaż, 
że w żadnej iteracji nie powstaje cykl. \\

\noindent Algorytm Boruvki w normalnej wersji działa tak, że dla każdego
wierzchołka grafu $G$ wybieramy krawędź z najmniejszą wagą i dodajemy ją
do zbioru $E^\prime$. Jeżeli po tym kroku zostanie kilka spójnych składowych,
zamieniamy je w \textit{superwierzchołki} (czyli sklejamy w jeden), 
a z pozostałymi krawędziamy postępujemy tak, jakbyśmy zaczynali algorytm od
nowa. Kroki te powtarzamy do momentu, aż w grafie pozostanie jedna spójna
składowa. \\

\noindent Mamy udowodnić, że w każdej iteracji (jak również po zakończeniu)
algorytmu Boruvki nie powstaje żaden cykl. Załóżmy nie wprost, że w którejś
iteracji algorytmu pojawia się spójna składowa, w której jest cykl, nazwijmy
ją $A$. Rozważmy dwie możliwości powstawania tego cyklu:
\begin{enumerate}
    \item Spójna składowa $A$ powstała przez połączenie dwóch superwierzchołków
    $v_1$, $v_2$, czyli do zbioru $E^\prime$ zostały dodane jakieś krawędzie
    $e_i, e_j$. Zauważmy, że krawędź $e_i$ była najlższejszą krawędzią
    incydentną do $v_1$, a więc cykl $C$ zawierający $e_i$ będzie miał mniejszą
    wagę, niż cykl zawierający krawędź $e_j$. Ale skoro dodaliśmy do $v_2$
    najlżejszą krawędź incydentną $e_j$, to cykl $C$ zawierający $e_j$ musi
    mieć mniejszą wagę, niż cykl zawierający $e_i$. Dochodzimy więc do
    sprzeczności, gdyż taka sytuacja nie jest możliwa.
    \item Spójna składowa $A$ powstała przez połączenie się kilku (trzech
    lub więcej) superwierzchołków. Podzielmy więc powstały cykl $C$ w taki
    sposób: wierzchołki $v_i$ dla $i = \{ 1, \ldots, l \}$ są superwierzchołkami
    w tym cyklu $C$, a $e_i$ dla $i = \{ 1, \ldots, l \}$ są krawędziami dodanymi
    w kolejnych iteracjach algorytmu. Oznacza to, że krawędź $e_1 = \{v_1, v_2\},
    e_2 = \{ v_2, v_3 \}, \ldots, e_l = {v_l, v_1}$. Biorąc pod uwagę sposób, 
    w jaki działa algorytm Boruvki, można stwierdzić, że aby powstał taki cykl $C$,
    musiałoby zachodzić 
    \[ w(e_1) < w(e_2) < \ldots < w(e_{l-1}) < w(e_l) < w(e_1) \]
    więc dochodzimy do sprzeczności, gdyż $w(e_1) \nless w(e_1)$.
\end{enumerate}

\noindent Udowodniliśmy więc, że algorytm Boruvki działa poprawnie i że w żadnej
jego iteracji nie powstaje cykl.


\newpage
\noindent \textbf{Zadanie 6} \newline
Jak zmodyfikować alogrytm Boruvki, by działał również w grafach, w których
jakieś krawędzie mają takie same wagi? \\

\noindent Aby algorytm ten działał w grafach z krawędziami, które mają takie same
wagi, musimy je w jakiś sposób rozróżnić. Poindeksujmy więc wszystkie
krawędzie od $i$ do $\abs{E}$, a następnie dla takich samych wag krawędzi
do wyboru, wybierajmy za każdym razem tę z mniejszym indeksem. Dzięki temu
krawędzie, których nie dodamy do $E^\prime$ przy pierwszym przejściu, zostaną
dodane do MST w trakcie łączenia superwierzchołków (czyli spójnych składowych). \\

\noindent W podobny sposób można również posortować leksykograficznie wszystkie
krawędzie: dla krawędzi $\{ a, b \}, \{ c, d\}$ o takich samych wagach, a więc 
$w(a,b) = w(c,d)$ (wierzchołki $a, b, c, d$ nie muszą być różne). Wtedy możemy
za każdym razem wybierać "mniejszą" krawędź na podstawie porządku leksykograficznego.

\noindent \newline \textbf{Zadanie 10} \newline
W pewnej grupie muzykujących osób Ania gra na skrzypcach, harfie, kontrabasie
i wiolonczeli, Bartek gra na harfie i fortepianie, Cezary gra na fortepianie,
Dąbrówka gra na harfie i Elwira gra na kontrabasie, skrzypcach, wiolonczeli i harfie.
Chcieliby zagrać utwór na fortepian, skrzypce, wiolonczelę, kontrabas i harfę.
Czy uda im się dobrać skład? \\

\noindent Przedstawmy skład i umiejętności wszystkich osób w tabelce:
\begin{table}[H]
    \centering
    \begin{tabular}{c|c|c|c|c|c|}
    \cline{2-6}
                                      & Ania & Bartek & Cezary & Dąbrówka & Elwira \\ \hline
    \multicolumn{1}{|c|}{skrzypce}    & X    &        &        &          & X      \\ \hline
    \multicolumn{1}{|c|}{harfa}       & X    & X      &        & X        & X      \\ \hline
    \multicolumn{1}{|c|}{kontrabas}   & X    &        &        &          & X      \\ \hline
    \multicolumn{1}{|c|}{wiolonczela} & X    &        &        &          & X      \\ \hline
    \multicolumn{1}{|c|}{fortepian}   &      & X      & X      &          &        \\ \hline
    \end{tabular}
\end{table}

\noindent Aby dobrać skład, musimy zacząć od osób, które potrafią grać tylko na
jednym instrumencie. Oznacza to, że Cezary musi grać na fortepianie, a Dąbrówka
na harfie. Wtedy Bartek niezbyt się przyda reszcie, gdyż nie potrafi grać na
żadnym z trzech brakujących instrumentów. Grupa zostaje więc z Anią i Elwirą
potrafiącą grać na brakujących skrzypcach, kontrabasie i wiolonczeli, jednak 
dwie osoby raczej sobie nie poradzą z grą na trzech instrumentach jednocześnie.
Oznacza to, że o ile Elwira i Ania nie są hiperuzdolnione, to nie uda się grupie
zagrać. \\

\noindent Można również zauważyć, że 
\[
    \abs{\text{Anna, Elwira}} < \abs{\text{wiolonczela, kontrabas, skrzypce}} 
\]
a więc nie jest spełniony warunek Halla, czyli się nie uda dobrać składu.

\end{document}